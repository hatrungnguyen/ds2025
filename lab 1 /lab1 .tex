\documentclass{article}
\usepackage{listings}
\usepackage{amsmath}

\title{File Transfer Using Sockets in Python}
\author{Your Name}
\date{\today}

\begin{document}

\maketitle

\section{Introduction}
In this report, we present a simple implementation of file transfer using Python's socket programming. The implementation consists of two parts: a client that sends a file and a server that receives the file.

\section{Client Implementation}
The client is responsible for establishing a connection to the server and sending the specified file. Below is the code for the client:

\begin{lstlisting}[language=Python]
import socket

def send_file(filename, host='127.0.0.1', port=65432):
    with socket.socket(socket.AF_INET, socket.SOCK_STREAM) as s:
        s.connect((host, port))
        with open(filename, "rb") as f:
            data = f.read(1024)
            while data:
                s.sendall(data)
                data = f.read(1024)
                print("send done")
        s.close()

if __name__ == "__main__":
    send_file("../1.txt")
\end{lstlisting}

\subsection{Code Explanation}
\begin{itemize}
    \item The code imports the \texttt{socket} library for network communication.
    \item The \texttt{send\_file} function takes the filename, host, and port as arguments.
    \item A socket is created and connected to the specified host and port.
    \item The file is opened in binary read mode, and data is sent in chunks of 1024 bytes using a loop.
    \item After sending the entire file, the connection is closed.
\end{itemize}

\section{Server Implementation}
The server listens for incoming connections and receives the file sent by the client. Below is the code for the server:

\begin{lstlisting}[language=Python]
import socket

def start_server(host='127.0.0.1', port=65432):
    with socket.socket(socket.AF_INET, socket.SOCK_STREAM) as s:
        s.bind((host, port))
        s.listen()
        print("Listen =))))")
        conn, add = s.accept()
        with conn:
            print(f"Connected by {add}")
            with open("../1.txt", "wb") as f:
                while True:
                    data = conn.recv(1024)
                    if not data:
                        break
                    f.write(data)
            print("File received")
        conn.close()

if __name__ == "__main__":
    start_server()
\end{lstlisting}

\subsection{Code Explanation}
\begin{itemize}
    \item The server code also imports the \texttt{socket} library.
    \item The \texttt{start\_server} function binds to a specified host and port, and listens for incoming connections.
    \item When a client connects, the server accepts the connection and begins receiving data.
    \item The received data is written to a file in binary write mode until no more data is sent.
    \item After the file transfer is complete, the connection is closed.
\end{itemize}

\section{Conclusion}
This implementation demonstrates a basic file transfer mechanism using Python's socket programming. The client sends a file to the server, which receives and saves it. This example can be expanded for more robust error handling, support for multiple clients, and secure data transfer.

\end{document}